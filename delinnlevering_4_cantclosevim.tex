\documentclass[12pt]{article}
\usepackage{graphicx}

\title{Delinnlevering 4 - cantclosevim}
\author{
    Joakim Tollefsen Johannesen\\
    \texttt{joakimtj@hiof.no}
    \and
    Niklas Berby\\
    \texttt{niklab@hiof.no}
}
\begin{document}

\maketitle

\newpage
\section{Plan for Brukertesting}
Fra siste brukertest er det en del endringer som er gjort for å forhåpentligvis skape en mer behagelig og enklere brukeropplevelse for de som
har null kjennskap til API'en.

Disse omhandler da det grunnleggende i API'et som for eksempel:
\begin{itemize}
    \item Instansiering av hovedressurser.
        \begin{itemize}
            \item Det `fysiske' konseptene som \texttt{Harbor}, \texttt{Ship*}, et al.
            \item Opprettelse av oppdrag i form av \texttt{Sailing}.
            \item Opprettelse og bruk av Logging i simulering.
        \end{itemize}
    \item Samspillet mellom objektene/klassene/typene.
    \item Kjøre enkle simuleringer uten å rote seg bort i den dypere funksjonaliteten til API'et.
\end{itemize}

Vi er da interessert om disse tiltakene har hjulpet, samt om de nye funksjonene som er implementert når en grad av vanskelighet som vi mener er
akseptabelt.
Den nye funksjonaliteten er som følger:
\begin{itemize}
    \item Bygge strukturene for simulering av frakt.
    \begin{itemize}
        \item \texttt{StorageArea}
        \item \texttt{StorageColumn}
        \item \texttt{Dock}
    \end{itemize}
    \item Opprette kjøretoy som sørger for bevegelse av frakt innenfor havnen.
    \begin{itemize}
        \item \texttt{AGV}
        \item \texttt{Crane}
    \end{itemize}
    \item Opprette fraktskipet og dets frakt.
    \begin{itemize}
        \item \texttt{Container}
    \end{itemize}
    \item Kjøre en enkel frakt simulering.
\end{itemize}

Brukertestingen vår er da hovedsakelig i to deler med en tredje om vi føler den er nødvendig for ytterligere informasjon om vi ser at det er store
misforståelser og hull i de to tidligere delene.

\subsection{Del 1} 

I del én er testgruppen gitt oppgaver som skal i utgangspunktet hjelpe dem å bygge en grunnleggende forståelse av API'et. 

Der det er relevant er de gitt en kort beskrivelse av konseptene. Som for eksempel om det er tvil i hva en havn ér.

Vi har planlagt følgende oppgaver der testgruppen har intellisens.
Det blir også gitt hint om de står stille eller føler de trenger det for å forstå oppgaven. Hintene er gitt rekkefølge fra minste nyttig til mest nyttig.

\subsubsection{Oppgave - Opprett en Havn}

En havn er et vannområde der skip og båter kan bli fortøyd. \\ \\
Opprett en havn og konfigurer den slik at skip kan fortøyes der. \\
Ta i bruk de klasser, metoder eller konstruktører dere føler er hensiktsmessige. \\ \\
Her skal testgruppen opprette en havn med en eller flere kaiplasser.

\subsubsection*{Hint}

\begin{enumerate}
    \item Havn er `Harbor'.
    \item Kaiplass er `Dock'.
    \item En havn kan ha flere kaiplasser.
    \item Benytt hjelpemetode for å lage vilkårlig antall kaiplasser.
\end{enumerate}

\subsubsection{Oppgave - Opprett et Skip}

Et skip er et fartøy som i kommersielt bruk har enten passasjerer ombord eller gods den frakter. \\ \\
Opprett et skip dere selv føler er best egnet til å senere utføre en enkel seiling. Det vil si frakt av gods ikke er nødvendig. \\ \\
Her skal testgruppen opprette et skip. Skipet kan være både av type \texttt{CommercialShip} og \texttt{FreightShip} da simuleringsoppgaven
ikke tar stilling til dette.

\subsubsection*{Hint}

\begin{enumerate}
    \item Skip er `\texttt{Ship}'.
    \item `\texttt{Ship}' er en abstrakt klasse.
    \item `\texttt{CommercialShip}' vil da være beste analog til et passasjer skip.
    \item `\texttt{CommercialShip}' vil ikke kunne brukes til frakt av gods.
\end{enumerate}

\subsubsection{Oppgave - Opprett en eller flere Seilinger}

En seiling er et `oppdrag' der et skip reiser til en kaiplass. Seilingen skjer i et spesifisert tidsrom - altså fra en start-tid til en slutt-tid. \\ \\
Opprett en eller flere seilinger med grunnlag i de tidligere oppgavene. \\ \\
Her skal testgruppen opprette seiling(er) med objektene de opprettet i de forrige oppgavene. 

\subsubsection*{Hint}

\begin{enumerate}
    \item Seiling er `\texttt{Sailing}'.
    \item `\texttt{Sailing} krever et skip.
    \item `\texttt{Sailing}' krever et reisemål.
    \item `\texttt{Sailing}' krever tidsrommet for reisen.
\end{enumerate}

\subsubsection{Oppgave - Opprett en Logger}

Det er flere typer loggere. Den mest enkle skriver kun relevante beskjeder under simuleringen til konsollen. \\ \\
Opprett en logger for senere bruk.

Her skal testgruppen opprette en logger som de skal bruke i simulerings klassen for å få utskrift av hendelser i simuleringen.

\subsubsection*{Hint}

\begin{enumerate}
    \item Loggere har LogLevel(s) som bestemmer hvilke meldinger som skal skrives ut.
    \item Verbøse-meldinger, Advarsels-meldinger etc.
    \item Simulering krever en logger.
\end{enumerate}

\subsubsection{Oppgave - Opprett og Kjør en Simulering}

En simulering kjører de oppdragene (seilingene) som er gitt. \\ \\
Opprett simulatoren og konfigurer tidligere objekter for å kunne kjøre en simulering. \\ \\
Den største oppgaven i fase 1. Det viktigste punktet i denne oppgaven er at de får med seg at seilinger er lagret i havnen som skal utføre den.
De må også lære at de selv må ta hensyn til om en kaiplass er opptatt og selv opprette seilinger der det ikke er noen konflikter
mellom forskjellige skips reiser for at de skal utføres.

\subsubsection*{Hint}

\begin{enumerate}
    \item Simulatoren heter `\texttt{SimulateHarbor}'
    \item Simulatoren krever en Logger for å kunne utføre simuleringen.
    \item Simulatoren utfører de seilingene i de havnene som er gitt som argument.
    \item Simulatoren utfører kun de seilingene som er innenfor det gitte tidsrommet til simuleringen.
\end{enumerate}

\newpage
\subsection{Del 2}

I del to skal testgruppen utføre en mer avansert seiling. Dette vil kreve en dypere forståelse av API'et da de må opprette ressurser
for konsepter de kanskje ikke har noen tidligere kunnskap om. Mer spesifikt: frakt av gods.

I del to får de tilgang til dokumentasjon og et klassediagram.

\subsubsection{Oppgave - Frakt: Havn Konfigurasjon}

Tilføy eller gjør endringer på havnene dere lagde for å støtte bevegelse av gods i havnen. \\ \\
Her skal testgruppen opprette StorageArea og StorageColumn(s). De skal også opprette AGV'er.

\subsubsection*{Hint}

\begin{enumerate}
    \item Havner har et område dedikert til lagring.
    \item Lagringsområdet har også flere steder der gods er lagret.
\end{enumerate}

\subsubsection{Oppgave - Frakt: Kaiplass Konfigurasjon}

Tilføy eller gjør endringer på kaiplassene til havnene for å støtte lossing av gods. \\ \\
Her skal testgruppen legge til et vilkårlig antall kraner og et vilkårlig antall AGV'er.

\subsubsection*{Hint}

\begin{enumerate}
    \item Lossing av gods håndteres av kraner.
    \item AGV'er flytter godset fra kaiplassen til lagringskolonnene.
\end{enumerate}

\subsubsection{Oppgave - Frakt: Fraktskip}

Opprett et skip og dets last. Lag seilingene som skal utføres.


\subsection{Del 3}

\section{Gjennomføring og Evaluering av Brukertesting}

\section{Evaluering av tildelt gruppes API}

JECH sitt rammeverk er en service-basert løsning. 



\section{Implementasjon av Grafisk Grensesnitt}

\end{document}